\documentclass[a4paper, 12pt]{article}

\usepackage[portuges]{babel}
\usepackage[utf8]{inputenc}
\usepackage{amsmath}
\usepackage{indentfirst}
\usepackage{graphicx}
\usepackage{multicol}

\begin{document}
%\maketitle

\begin{titlepage}
	\begin{center}
	
	\begin{figure}[!ht]
	\centering
	\includegraphics[width=2cm]{relatorio/ufsj.png}
	\end{figure}

		\LARGE{Universidade Federal de São João del Rei}\\
		\large{Departamento de Ciência da Computação}\\ 
		\large{Sistemas Operacionais}\\
		\large{\textbf{Professor:} Rafael Sachetto} \\
		\vspace{15pt}
        \vspace{95pt}
        \textbf{\LARGE{Relatório Trabalho Prático 1}}\\
		%\title{{\large{Título}}}
		\vspace{3,5cm}
	\end{center}
	
	\begin{flushright}
   \begin{list}{}{
      \setlength{\leftmargin}{4.5cm}
      \setlength{\rightmargin}{0cm}
      \setlength{\labelwidth}{0pt}
      \setlength{\labelsep}{\leftmargin}}

      \item Relatório do primeiro Trabalho Prático para a disciplina de Sistemas Operacionais do Bacharelado em Ciência da Computação da Universidade Federal de São João del Rei.

      \begin{list}{}{
      \setlength{\leftmargin}{0cm}
      \setlength{\rightmargin}{0cm}
      \setlength{\labelwidth}{0pt}
      \setlength{\labelsep}{\leftmargin}}

			\item \hfill Filipe Mateus \
            \item \hfill Gustavo Detomi \
      		\item \hfill Marcos Martins \

      \end{list}
   \end{list}
 \end{flushright}
	\vspace{1cm}
	
	\begin{center}
		\vspace{\fill}
			 Maio\\
		 2022
			\end{center}
\end{titlepage}

\newpage

\tableofcontents
\thispagestyle{empty}

\newpage
\pagenumbering{arabic}

\section{Introdução}
Para este Trabalho Prático foi implementado um interpretador de comandos que foi 
chamado \textbf{\textit{shellso}}, e conta com diversas 
funcionalidades comuns aos interpretadores de comandos: disparar processos, 
encadear a comunicação entre eles usando pipes, executar processos em segundo 
plano, listar processos disparados pelo interpretador, retornar processos do 
segundo plano de volta para o interpretador, entre outras.



\newpage

\section{Resumo do projeto}
Algumas estruturas importantes:

\begin{verbatim}
struct shell {
    bool running;
    bool verbose;
    prompt_function prompt;
    aliasses* aliasses;
    background_jobs* jobs;
    shell_builtin_commands* builtin_commands;
}
\end{verbatim}


\begin{verbatim}
struct command {
    int argc;                      
    char** argv;                    
    command* next;                  
    command_chain_type chain_type;  
    char* stdin_file_redirection;    
    char* stdout_file_redirection;   
    char* stderr_file_redirection;   
}
\end{verbatim}

\begin{verbatim}
struct builtin_command {
    char* command;
    builtin_command_function function;
}
\end{verbatim}

\begin{verbatim}
struct _shell_builtin_commands {
    builtin_command** commands;
    int size;
    int count;
} shell_builtin_commands;
\end{verbatim}

\begin{verbatim}
struct background_job {
    char* command;
    int pid;
    int status;
    background_job* next;
}
\end{verbatim}

\newpage

\section{Decisões de projeto}
\newpage

\section{Bugs conhecidos ou problemas}
\newpage

\addcontentsline{toc}{section}{Anexo}
\section*{Anexo}
Repositório: https://github.com/marcosfons/shellso
Documentação online: https://marcosfons.github.io/shellso/
\end{document}

