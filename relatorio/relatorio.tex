\documentclass[a4paper, 12pt]{article}

\usepackage[portuges]{babel}
\usepackage[utf8]{inputenc}
\usepackage{amsmath}
\usepackage{indentfirst}
\usepackage{graphicx}
\usepackage{multicol}

\begin{document}
%\maketitle

\begin{titlepage}
	\begin{center}
	
	\begin{figure}[!ht]
	\centering
	\includegraphics[width=2cm]{relatorio/ufsj.png}
	\end{figure}

		\LARGE{Universidade Federal de São João del Rei}\\
		\large{Departamento de Ciência da Computação}\\ 
		\large{Sistemas Operacionais}\\
		\large{\textbf{Professor:} Rafael Sachetto} \\
		\vspace{15pt}
        \vspace{95pt}
        \textbf{\LARGE{Relatório Trabalho Prático 1}}\\
		%\title{{\large{Título}}}
		\vspace{3,5cm}
	\end{center}
	
	\begin{flushright}
   \begin{list}{}{
      \setlength{\leftmargin}{4.5cm}
      \setlength{\rightmargin}{0cm}
      \setlength{\labelwidth}{0pt}
      \setlength{\labelsep}{\leftmargin}}

      \item Relatório do primeiro Trabalho Prático para a disciplina de Sistemas Operacionais do Bacharelado em Ciência da Computação da Universidade Federal de São João del Rei.

      \begin{list}{}{
      \setlength{\leftmargin}{0cm}
      \setlength{\rightmargin}{0cm}
      \setlength{\labelwidth}{0pt}
      \setlength{\labelsep}{\leftmargin}}

			\item \hfill Filipe Mateus \
            \item \hfill Gustavo Detomi \
      		\item \hfill Marcos Martins \

      \end{list}
   \end{list}
 \end{flushright}
	\vspace{1cm}
	
	\begin{center}
		\vspace{\fill}
			 Maio\\
		 2022
			\end{center}
\end{titlepage}

\newpage

\tableofcontents
\thispagestyle{empty}

\newpage
\pagenumbering{arabic}

\section{Introdução}
Para este Trabalho Prático foi implementado um interpretador de comandos que foi 
chamado \textbf{\textit{shellso}}, e conta com diversas 
funcionalidades comuns aos interpretadores de comandos: disparar processos, 
encadear a comunicação entre eles usando pipes, executar processos em segundo 
plano, listar processos disparados pelo interpretador, retornar processos do 
segundo plano de volta para o interpretador, entre outras.





\section{Resumo do projeto}
Algumas estruturas importantes:

\begin{verbatim}
struct shell {
    bool running;
    bool verbose;
    prompt_function prompt;
    aliasses* aliasses;
    background_jobs* jobs;
    shell_builtin_commands* builtin_commands;
}
\end{verbatim}

A estrutura \textit{shell} representa a base do nosso interpretador de comandos, e deve manter
registro dos jobs, dos comandos próprios e dos \textit{aliasses}.

\begin{verbatim}
struct command {
    int argc;                      
    char** argv;                    
    command* next;                  
    command_chain_type chain_type;  
    char* stdin_file_redirection;    
    char* stdout_file_redirection;   
    char* stderr_file_redirection;   
}
\end{verbatim}

A estrutura \textit{shell} representa os comandos em si, possui os campos necessários para
armazenar informações do comando, do encadeamento de comandos e redirecionamento 
de entrada e saída. 

\begin{verbatim}
struct builtin_command {
    char* command;
    builtin_command_function function;
}
\end{verbatim}

A estrutura \textit{builtin\_command} é usada para representar os comandos implementados
internamente no \textbf{\textit{shellso}}.

\begin{verbatim}
struct _shell_builtin_commands {
    builtin_command** commands;
    int size;
    int count;
} shell_builtin_commands;
\end{verbatim}

É uma tabela \textit{Hash} contendo todos os comandos que foram implementados diretamente
no \textit{shell}.

Alguns exemplos de \textit{builtin\_commands} são \textit{jobs} e \textit{fg}. Esses comandos
são necessários para controlar a execução de processos filhos do interpretador. O comando 
\textit{jobs} percorre os \textit{backgroud\_jobs} do shell mostrando informações relevantes sobre eles,
se estão executando, se foram terminados com sucesso além do \textit{pid} de cada um. Enquanto isso
o \textit{buitin\_command} \textit{fg} traz de volta para o shell aqueles processos que estavam em segundo
plano. Abaixo temos um exemplo do uso deles:

\begin{figure}[!ht]
	\centering
	\includegraphics[width=12cm]{relatorio/jobs.jpeg}
\end{figure}

Nesse exemplo foi usado o \textit{shellso} para executar o vim,
em seguida sua execução foi pausada e isso pode ser verificado com
o \textit{jobs}, após executar um \textit{ls} o \textit{jobs} mostra
o processo do \textit{vim} pausado e o processo do \textit{ls} terminado com sucesso.
Após isso foi executado outro \textit{builtin\_command} \textit{fg}
para trazer o \textit{vim} para o primeiro plano e terminar sua execução.
Após isso o \textit{jobs} pode ser utilizado uma última vez para verificar que o processo foi de fato terminado.


\begin{verbatim}
struct background_job {
    char* command;
    int pid;
    int status;
    background_job* next;
}
\end{verbatim}

Essa estrutura representa os comandos que estão sendo executados em segundo plano pelo \textbf{\textit{shellso}}.
Um caso de uso interessante dessa estrutura foi no \textit{handler} do sinal \textit{SIGCHLD}.
Nessa função foi preciso identificar quais processos de segundo plano foram finalizados, e
quais continuam em execução, para isso é percorrida a \textit{lista encadeada} de 
textit{background\_job}'s do \textit{shell}, para cada um usamos a chamada de sistema 
\textit{waitpid()} passando a flag WNOHANG, essa flag faz com que a chamada
retorne imediatamente o estado de um determinado processo filho sem
esperar até que ele seja terminado.



\section{Decisões de projeto}

O \textbf{\textit{shellso}} foi projetado de forma a ser dividido em módulos bem separados
com o objetivo de tornar seu desenvolvimento e eventual manutenção mais fácil. Cada elemento
do interpretador foi implementado em arquivos separados dentro da pasta \textit{shell}, algumas
funções utilitárias para trabalhar com cadeias de caracteres foram implementadas na pasta \textit{string}.

Essa decisão de dividir o programa em módulos primeiramente divide bem as responsabilidades, 
o que garante que a pessoa desenvolvedora possa trabalhar em um pedaço pequeno do problema simplificando
o projeto e a implementação, além disso auxilia a evitar efeitos colaterais, já que se ao
longo de todo o problema as estruturas apresentadas na seção anterior fossem visíveis e 
alteráveis seria bastante provável que ao longo do desenvolvimento e eventuais manutenções
fossem provocados efeitos colaterais ou seja, alterações introduzidas sem cuidado
causando modificações incompletas nas estruturas de dados, provocando erros difíceis de reproduzir e detectar.

Outra decisão que tomamos foi de implementar testes, para isso usamos a blibioteca Criterion, disponível em 
\textit{https://criterion.readthedocs.io/en/master/intro.html}. Com essa simples biblioteca de testes 
foi possível escrever testes de unidades que facilitam o desenvolvimento tornando mais fácil
visualizar quando um código novo resolve o problema a que se propõe sem causar efeitos colaterais
em outras partes do código.



\section{Bugs conhecidos ou problemas}

Um erro conhecido desse trabalho é relacionado aos redirecionamentos
de entrada e saída. A cada \textit{pipe} utilizado um \textit{file descriptor} é deixado aberto,
não foi possível identificar onde seria necessário fechar esse descritor de arquivo. Abaixo temos um exemplo desse bug:

\begin{figure}[!ht]
	\centering
	\includegraphics[width=12cm]{relatorio/fd.jpeg}
\end{figure}

Um recurso interessante que não foi implementado foi a captura de alguns atalhos
que poderiam facilitar o uso do interpretador, como o \textit{ctrl+L} para limpar
o terminal e das teclas direcionais para buscar comandos recém executados ou para
movimentar o cursor do prompt para corrigir um comando sem reescrevê-lo do início.




\addcontentsline{toc}{section}{Anexo}
\section*{Anexo}
Repositório: https://github.com/marcosfons/shellso

Documentação online: https://marcosfons.github.io/shellso/
\end{document}

