\documentclass[a4paper, 12pt]{article}

\usepackage[portuges]{babel}
\usepackage[utf8]{inputenc}
\usepackage{amsmath}
\usepackage{indentfirst}
\usepackage{graphicx}
\usepackage{multicol}

\begin{document}
%\maketitle

\begin{titlepage}
	\begin{center}
	
	%\begin{figure}[!ht]
	%\centering
	%\includegraphics[width=2cm]{c:/ufba.jpg}
	%\end{figure}

		\LARGE{Universidade Federal de São João del Rei}\\
		\large{Departamento de Ciência da Computação}\\ 
		\large{Sistemas Operacionais}\\
		\large{\textbf{Professor:} Rafael Sachetto} \\
		\vspace{15pt}
        \vspace{95pt}
        \textbf{\LARGE{Relatório Trabalho Prático 1}}\\
		%\title{{\large{Título}}}
		\vspace{3,5cm}
	\end{center}
	
	\begin{flushright}
   \begin{list}{}{
      \setlength{\leftmargin}{4.5cm}
      \setlength{\rightmargin}{0cm}
      \setlength{\labelwidth}{0pt}
      \setlength{\labelsep}{\leftmargin}}

      \item Relatório do primeiro Trabalho Prático para a disciplina de Sistemas Operacionais do Bacharelado em Ciência da Computação da Universidade Federal de São João del Rei.

      \begin{list}{}{
      \setlength{\leftmargin}{0cm}
      \setlength{\rightmargin}{0cm}
      \setlength{\labelwidth}{0pt}
      \setlength{\labelsep}{\leftmargin}}

			\item \hfill Filipe Mateus \
            \item \hfill Gustavo Detomi \
      		\item \hfill Marcos Martins \

      \end{list}
   \end{list}
 \end{flushright}
	\vspace{1cm}
	
	\begin{center}
		\vspace{\fill}
			 Maio\\
		 2022
			\end{center}
\end{titlepage}
%%%%%%%%%%%%%%%%%%%%%%%%%%%%%%%%%%%%%%%%%%%%%%%%%%%%%%%%%%%

% % % % % % % % %FOLHA DE ROSTO % % % % % % % % % %

\newpage
% % % % % % % % % % % % % % % % % % % % % % % % % %
\newpage
\tableofcontents
\thispagestyle{empty}

\newpage
\pagenumbering{arabic}
% % % % % % % % % % % % % % % % % % % % % % % % % % %
\section{Introdução}
\newpage
\section{Resumo do projeto}
\newpage

\section{Decisões de projeto}

\newpage
\section{Bugs conhecidos ou problemas}
\newpage


\addcontentsline{toc}{section}{Anexo}
\section*{Anexo}
https://github.com/marcosfons/shellso
\end{document}

